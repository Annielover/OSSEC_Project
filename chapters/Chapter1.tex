\chapter{Introduction}
\emph{In this chapter we will provide some general information about intrusions and security problems.}
\minitoc

\section{Intrusions}

An important aspect of the security in networking is to prevent attacks form happening in the first place and the ability to fend off possible attackers. Therefore, intrusion detection systems have been developed \citep{IDS}. \\ \\
First, we will define what an intrusion actually is, before digging into intrusion detection systems. According to \citep{tr} intrusions are actions that compromise the availability, integrety and confidentiality of computer systems or any attempts to do so. That is,  the attempts or actions of unauthorized entry into an computer system or network. This action ranges from a reconnaissance attempt to map any existence of vulnerable services, real attacks and finally the embedding of backdoors. Such process can for example result in the creation of an illegal account with administrator privilege upon the victim machine \citep{76}. 

Needless to say, using this account, the attacker can entirely control the victim's machine and this should be prevented at any price.

\subsection{Intrusion Prevention Systems}

There have been several approaches or technologies designed to prevent such unwanted actions. We call them Intrusion Prevention Systems or IPSs. Some examples of prevention approaches or systems that exists today include antivirus, strong authentication, cryptography, patch management and firewalls.

When an attack is identified, intrusion prevention block and log the offending data.

Antivirus systems exist to prevent malicious programs such as virusses, worms and trojan horses from successfully being embedded or executed within a system. Patch management ensures deployment of the latest security patches so as to prevent system vulnerabilities from successfully exploited.

Crypography is used to prevent any attempt to compromise sensitive information. Finally, strong authentication exists to prevent any attempts to fake an identity in an effort to enter a particular system \citep{76}. 

However, firewalls provide a prevention measure up until the application layer\footnote{Application layer of the OSI model: the application layer is the very top layer of the OSI model. It is used by network applications. These applications are what actually implement the functions performed by users to accomplish various tasks over the network. Examples include FTP, DNS and HTTP \citep{osi}.}. But this measure is commonly implemented only to port number and IP address while various intrusion attempts are intelligently exploiting vulnerability in applications which are opened by the firewall. Therefore, we need a newer and more sophisticated system to subdue these shortcomings. We call this system an intrusion prevention system or IPS.

\section{General security problems overview}

Having defined intrusions, we can now dig deeper into how a cracker breaks into a private system or network. Unfortunately, there are many ways to do so and such an action is usually not a one-shot attempt. \\ \\
One can devide the `intrusion life cycle' into three phases:
\begin{enumerate}
\item \textbf{Information gathering:} in the first phase, an attacker tries to gain / discover as much as information that is possible about the target computer system or network. This can for example be achieved using a port scan. Information sought by the hacker consists of DNS tables, open ports (that's why one needs a firewall), available hosts, OS type, etc\ldots. The information collected in this phase will eventually determine the type of attack, which is described in the next item.
\item \textbf{The actual attack:} when a hacker has collected enough information about the target system of network, he can begin initiating the attack. Therefore, various attack techniques exist, including password brute force attempts, buffer overflows, spoofing, etc\ldots. When the attacker has successfully broken in into a network, he will be able to gain control (or to crash) of the target system with all possible further consequences, including service disruption.
\item \textbf{Profileration:} now that the hacker has gained access to the network, he can obtain sensitive or valuable information. Examples include copying files, perform queries on databases and obtain general network information. The attacker also ensures he can come back anytime to this compromised system by, for example, modifying firewall filtering rules or by installing a Trojan. This is done to use this compromised system/network to launch attacks against other (private) systems or networks. One shall obviously be familiar with the term DoS\footnote{DoS: a denial-of-service (DoS) attack is aimed at blocking availability of computer systems or services. It is an action (or set of actions) executed by a malicious entity to make a resource unavailable to its intended users \citep{Dos}.} or DDoS \footnote{DDoS: same as DoS, but the distributed format adds the ``many to one'' dimension that makes these attacks more difficult to prevent. First, it involves the target host that has been chosen to receive the brunt of the attack. Second, it involves the presence of multiple attack daemon agents. These are agent programs that actually conduct the attack on the target victim. Attack
daemons are usually deployed in host computers. The third component of a distributed denial of service attack is the control master program. Its task is to coordinate the attack.
Finally, there is the real attacker, the mastermind behind the attack. By using a control master program, the real attacker can stay behind the scenes of the attack \citep{Ddos}.}.
\end{enumerate}

In this paper, we present an introduction of intrusion detection / prevention systems as well as an in-depth study of Snort, an open source network prevention and detection program \citep{Snort}.